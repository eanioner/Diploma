% Формат бумаги: А4.
\documentclass[a4paper]{report}
\usepackage[utf8]{inputenc}
\usepackage[russian]{babel}
\usepackage{amsmath,amsfonts,amssymb,cite,url,verbatim,graphicx,wrapfig}
\graphicspath{{images/}}

% Поля: верхнее – 2 см, нижнее – 2 см, левое – 3 см, правое – 1.5 см.
\usepackage{geometry}
\geometry{
	left = 3cm,
	top = 2cm,
	right = 1.5cm,
	bottom = 2cm
}


\setlength{\parindent}{1.5cm}

% Кегль: основной текст – 14 пт, названия параграфов – 16 пт, названия глав – 18 пт, текст в таблице, подписи к рисункам, таблицам – 12 пт.
\renewcommand{\small}{\fontsize{12}{12}\selectfont}
\renewcommand{\normalsize}{\fontsize{14}{14}\selectfont}
\renewcommand{\large}{\fontsize{16}{16}\selectfont}
\renewcommand{\Large}{\fontsize{18}{18}\selectfont}
\renewcommand{\huge}{\fontsize{20}{20}\selectfont}
\usepackage{sectsty}
\sectionfont{\Large}
\subsectionfont{\large}
\paragraphfont{\normalsize}
\usepackage[font=small]{caption}

% Межстрочный интервал: 1.5 строки.
\usepackage{setspace}
\onehalfspacing

% Абзацный отступ. Первая строка каждого абзаца должна иметь абзацный отступ 1.25 см.
\usepackage{indentfirst}
\setlength{\parindent}{1.25cm}

% Выравнивание основного текста по ширине поля.
\usepackage{ragged2e}
\justifying

%%% здесь лучше ничего не трогать
\setcounter{tocdepth}{2}
\renewcommand{\thesection}{}
\renewcommand{\thesubsection}{}
\allsectionsfont{\centering}
\usepackage{titlesec}
\newcommand{\sectionbreak}{\clearpage}
\usepackage{tocloft}
\cftsetindents{section}{0em}{0em}
\cftsetindents{subsection}{2em}{0em}
\AtBeginDocument{\renewcommand{\contentsname}{\begin{center} \vskip-2.5cm \Large{Содержание} \end{center}}}
\AtBeginDocument{\renewcommand{\bibname}{}}
%%%

\begin{document}
	%%%%% TITLE PAGE %%%%%
\thispagestyle{empty}
\begin{center}
\textsc{Санкт-Петербургский государственный университет \\
\textbf{Кафедра технологий программирования}} \\
\vspace{1cm}
\Large{\textbf{Башарин Егор Валерьевич}} \\
\vspace{1cm}
\large{\textbf{Выпускная квалификационная работа бакалавра}} \\ 
\vspace{2cm}
\Large{\textbf{Контекстный анализ данных социальных сетей}} \\ 
\normalsize{010400 \\
Прикладная математика и информатика} \\
\end{center}
\vspace{2cm}
\hspace{9cm} Научный руководитель, \\
\hspace*{9cm} старший преподаватель \\
\hspace*{9cm} Попова С.В. 
\begin{center}
\vfill
Санкт-Петербург \\
2016
\end{center}
	\newpage
	
	%%%%% CONTENTS %%%%%
	\tableofcontents
	\newpage
	
	
%%%%%%%%%%%%%%%%%%%%%%%%%%%%%%%%%%%%%%%%%%%%%%%%%%%%%%%
%%%%%%%%%%%%%%%%%%%%%%%%%%%%%%%%%%%%%%%%%%%%%%%%%%%%%%%
                   %%%%% Intro %%%%%
%%%%%%%%%%%%%%%%%%%%%%%%%%%%%%%%%%%%%%%%%%%%%%%%%%%%%%%
%%%%%%%%%%%%%%%%%%%%%%%%%%%%%%%%%%%%%%%%%%%%%%%%%%%%%%%
	\section{Введение} 
	В настоящее время явление социальных сетей достаточно распространено. Социальные сети уверенно вошли в жизнь современного человека и теперь занимают в ней значимую часть. Главным образом они оказывают влияние на поведение, предубеждения, ценности и намерения человека, что отражается во всех сферах его деятельности. Оказываемое влияние, быстрый рост популярности и открытый доступ к контенту привлекли к социальным сетям внимание правительства, финансовых организации и исследователей. Преобразование контента социальных сетей в текстовую информацию и выделение ключевых концепций стало важным условием для порождения знаний и формулирования стратегий. Анализ полученной информации помогает исследователям улучшить понимание об информационных потоках, о формировании и распространении мнений, о связи ценностей и предубеждений пользователя и генерируемого им контента. Тем не менее число качественных и количественных исследований в данной области слишком мало. Самый значительный барьер при использовании социальных сетей - это отсутствие методологии для выбора, сбора, обработки и анализа информации, полученной с сайтов социальных сетей. Однако, существуют компании по производству программного обеспечения, разработавшие проприетарные системы сбора информации для визуализации данных, и исследователи, разработавшие экспертные системы для анализа настроений \cite{bib:Kaklauskas}. 
	
Пользователи социальных сетей активно публикуют данные о своей ежедневной активности, чувствах и мыслях, выражая свое мнение и позицию. Благодаря этому в социальных сетях образуются группы пользователей (сообщества), имеющих общие интересы. Для выявления ключевых концепций и тематик присущих группе пользователей используется контекстная обработка данных, генерируемых участниками группы. В данной работе контекстная обработка данных основана на идеях и принципах тематического моделирования. Результаты такой обработки данных могут использоваться для мониторинга заболеваний, например гриппа \cite{bib:Paul}, или для предсказания поведения рынка. \\

%%%%%%%%%%%%%%%%%%%%%%%%%%%%%%%%%%%%%%%%%%%%%%%%%%%%%%%
%%%%%%%%%%%%%%%%%%%%%%%%%%%%%%%%%%%%%%%%%%%%%%%%%%%%%%%
                   %%%%% Постановка задачи %%%%%
%%%%%%%%%%%%%%%%%%%%%%%%%%%%%%%%%%%%%%%%%%%%%%%%%%%%%%%
%%%%%%%%%%%%%%%%%%%%%%%%%%%%%%%%%%%%%%%%%%%%%%%%%%%%%%%
	 
	\section{Постановка задачи}
	Целью данной работы является изучение методов контекстной обработки данных социальных сетей, в основе которых лежат при принципы и идеи тематического моделирования. Для того чтобы достичь поставленной цели предлагается выполнить следующий ряд задач:
	
	
	\renewcommand{\labelenumi}{\arabic{enumi}.}
	\renewcommand{\labelenumii}{\arabic{enumi}.\arabic{enumii}}

	\begin{enumerate}
	\item{Анализ предметной области}
		\begin{enumerate}
		\item{Обзор социальных сетей}
		\item{Выбор социальной сети в качестве источника данных}
		\end{enumerate}
	\item{Подготовка данных}
		\begin{enumerate}
		\item{Загрузка данных с веб-страниц социальной сети}
		\item{Предобработка данных}
		\item{Разбиение данных на обучающую и тестовую части}
		\end{enumerate}
	\item{Выбор тематической модели}
		\begin{enumerate}
		\item{Анализ тематических моделей}
		\item{Выбор подходящей тематической модели}
		\end{enumerate}
	\item{Построение тематической модели}
		\begin{enumerate}
		\item{Анализ методов построения тематической модели}
		\item{Реализация программного модуля, реализующего выбранную тематическую модель}
		\item{Обучение тематической модели}
		\end{enumerate}
	\item{Оценка качества модели}
		\begin{enumerate}
		\item{Обзор и анализ оценок качества тематических моделей}
		\item{Оценка качества построенной модели}
		\end{enumerate}
	\item{Анализ полученных результатов}
	\end{enumerate}

%%%%%%%%%%%%%%%%%%%%%%%%%%%%%%%%%%%%%%%%%%%%%%%%%%%%%%%
%%%%%%%%%%%%%%%%%%%%%%%%%%%%%%%%%%%%%%%%%%%%%%%%%%%%%%%
                   %%%%% Обзор литературы %%%%%
%%%%%%%%%%%%%%%%%%%%%%%%%%%%%%%%%%%%%%%%%%%%%%%%%%%%%%%
%%%%%%%%%%%%%%%%%%%%%%%%%%%%%%%%%%%%%%%%%%%%%%%%%%%%%%%
	\section{Обзор литературы}
	
	Цель данной работы тесно пересекается с информационным поиском, основы которого подробно рассмотрены в книге Кристофера Майнинга "Introduction to Information Retrieval" \cite{bib:InformationRetrieval}. Особое внимание стоит удалить главам 2 и 18. В главе~2 описываются методы подготовки и предобработки текстовой информации. Глава~18 сосредотачивает внимание на подходах латентно-семантического анализа, который является ценным инструментом в тематическом моделировании. В конце каждой главы приведены ссылки на литературу для более подробного изучения темы. 
	
	Вероятностное латентно-семантическое моделирование стало логичным продолжением идей латентно-семантического моделирования и нашло свое применение в тематическом моделировании. Это стало причиной появления вероятностных тематических моделей. Основные принципы вероятностного латентно-семантического анализа (probabilistic latent semantic analysis - pLSA) были описаны Томасом Хоффманом в 1999 году в статье \cite{bib:Hoffman} и были развиты Дэвидом Блеем в его статье 2005 года \cite{bib:Blei}, в которой была введена и рассмотрена тематическая модель латеного размещения Дирихле (latent dirichlet allocation - LDA). Статья Д.Блея описывает основные преимущества LDA перед pLSA, а так же методы построения и оценки качества тематической модели LDA. В статье Д.Блея \cite{bib:Blei2} 2012 года рассматриваются связь LDA с другими вероятностными тематическими моделями, а так же применение LDA в тематическом моделировании.
	
	В техническом отчете Грегора Хейнрича "Parameter estimation for text analysis" \cite{bib:Heinrich}  рассматриваются методы оценки параметров моделей для тематического анализа текстов. В отчете подробно разобраны темы, связанные с основными подходами для оценки параметров, сопряженными распределениями и Байесовскими сетями. А так же применение данных тем для построения тематической модели LDA. 
	
	 Среди русскоязычной литературы следует обратить внимание на работы К.~В.~Воронцова.  В работе \cite{bib:Voron1} подробно описаны основные идеи вероятностного тематического моделирования. В первой части данной работы ставится задача тематического моделирования. Далее рассматриваются основные вероятностные тематические модели PLSA, LDA и их модифицированные версии, а так же методы их построения. Работа завершается обзором способов оценки качества тематических моделей и обзором экспериментов с тематическими моделями. 
	

	
	
	
	\section{Глава 1. Подготовка данных}
	\subsection{1.1 Обзор социальных сетей}
	\subsection{1.2 Выбор социальной сети и получение данных}
	\subsection{1.3 Предобработка данных}
	\subsection{1.4 Результаты} 
	
	\section{Глава 2. Выбор и построение тематической модели}
	\subsection{3.1 Тематическое моделирование}
	\subsection{3.1 Выбор тематической модели}
	\subsection{3.2 Описание тематической модели LDA}
	\subsection{3.3 Реализация тематической модели}
	\subsection{3.4 Эксперименты}
	
	\section{Глава 3. Оценка качества модели}
	\subsection{4.1 Обзор оценок качества тематических моделей}
	\subsection{4.2 Оценка качества построенной тематической модели}
	
	\section{Анализ результатов}
	
	
	%%\ref{fig:image1}
	
	%%Вставим картиночку и сошлёмся на неё. \ref{fig:image1}
	%%\begin{figure}[h]
	%%	\centering
	%%	\includegraphics[width=\textwidth]{eyeinjection.png}
	%%	\caption{Это подпись к картиночке \label%%%%%{fig:image1}}
	%%\end{figure}
	
	
	
	
	\section{Заключение}
	
	
	
	\newpage
	\section{Список литературы}
	
    \begingroup
    \let\clearpage\relax
    \vskip-3cm
	\begin{thebibliography}{9}
	
		\bibitem{bib:Kaklauskas}
		Arturas Kaklauskas. Biometric and Intelligent Decision Making Support // Springer, 2015, P. 220.
		
		\bibitem{bib:Paul}
		Paul, MJ. and M. Dredze. You Are What You Tweet: Analyzing Twitter for Public Health. // In Proc. of the 5th International AAAI Conference on Weblogs and Social Media (ICWSM),  2011. 
		
		\bibitem{bib:InformationRetrieval}
		Christopher D. Manning, Prabhakar Raghavan and Hinrich Schütze. Introduction to Information Retrieval. Cambridge University Press, 2008. 506 P.
		%\bibitem{bib:DataMining}
		%Pang-Ning Tan, Michael Steinbach and Vipin Kumar. Introduction to Data Mining. Addison-Wesley, 2006. 769 P.	
		%\bibitem{bib:Seragan}
		%Toby Segaran. Programming Collective Intelligence. O'Reilly Media, 2007. 362 P.
		
		\bibitem{bib:Hoffman}
		Thomas Hofmann. Probabilistic latent semantic indexing. Proceedings of the Twenty-Second Annual International SIGIR Conference, 1999.
		
		\bibitem{bib:Blei}
		David M. Blei, Andrew Y. Ng, Michael I. Jordan. Latent Dirichlet Allocation // Journal of Machine Learning Research 3, 2003. P.~993~--~1022.
		
		\bibitem{bib:Heinrich}
		Gregor Heinrich. Parameter estimation for text analysis. Technical
report, Fraunhofer IGD, Darmstadt, Germany, 2005.

		\bibitem{bib:Blei2}
		David Blei. Introduction to Probabilistic Topic Models // Communications of the ACM, 2012. P. 77–84.

		\bibitem{bib:Voron1}
		Воронцов К.В. Вероятностное тематическое моделирование // www.machinelearning.ru : web. — 2013.

		
		\bibitem{bib:ml}
		ссылка на machine learning
		
		\bibitem{bib:vkapi}
		ссылка на апи вк
		
		\bibitem{bib:doc}
		можно всякие документации добавить

	
		
		

		
		
		%%% Примеры
		%%% КНИГА ОДНОГО АВТОРА
		%\bibitem{bib:Viner}
		%Винер~В. Нелинейные задачи в теории случайных процессов.~М.:~ИЛ,~1961. 159~с.
		
		%%% КНИГА НЕСКОЛЬКИХ АВТОРОВ
		%\bibitem{bib:Demidovich}
		%Демидович~Б.\:П., Марон~И.\:А., Шувалова~Э.\:З. Численные методы анализа. Приближение функций, дифференциальные и интегральные уравнения.~М.:~Наука,~1967. 368~c.
		
		%%% СТАТЬЯ В ЖУРНАЛЕ
		%\bibitem{bib:Nadaraya} 
		%Надарая~Э.\:А. Об оценке регрессии // Теория вероятностей и ее применения, 1964. Т.~9, Вып.~1. С.~157~--~159.
		
		%\bibitem{bib:Sinitzin}
		%Синицын~И.\:Н. Методы статистической линеаризации (обзор) // АиT,~1974. №~5. С.~82~--~94.
		
		%\bibitem{bib:Billings}
		%Billings~S.\:A., Fadzil~M.\:B., Sulley~J., Johnson~P.\:M. Identification of a nonlinear difference equation model of an industrial diesel generator // Mechanical Systems and Signal Processing,~1988. Vol.~2, No~1. P.~59~--~76.
		
		%\bibitem{bib:Bouton}
		%Booton~R.\:C. Nonlinear control systems with random inputs // Trans. IRE Profes. Group on Circuit Theory,~1954. Vol.~CT1, No~1. P.~9~--~18.
		
		%\bibitem{bib:Boyd}
		%Boyd~S., Chua~L.\:O. Fading memory and the problem of approximating nonlinear operators with Voltterra series // IEEE Trans. Circuits Syst.,~1985. Vol.~CAS-32, No~11. P.~1150~--~1161.
		
		%\bibitem{bib:Gagarin}
		%Garain~U. Identification of mathematical expressions in document images // Proc. of the 10th Int. Conf. on Document Analysis and Recognition (ICDAR),~2009. P.~1340~--~1344.
		
		%\bibitem{bib:Lee}
		%Lee~H.\:J., Wang~J.\:S. Design of a mathematical expression understanding system // Pattern Recognition Letters,~1997. Vol.~18, No~3, P.~289~--~298.
		
		
		
		%\bibitem{Steinberg}
		%Штейнберг~Ш.\:Е. Идентификация в системах управления.~М.:~Энергоатомиздат,~1987. 80~с.
		
		% \bibitem{bib:Petrov}
		%Петров~Л.\:И. О культуре набора текста на ПМ–ПУ.~СПб:~Изд. дом C.-Петерб. ун-та,~1726. 324~c.
		
		
		
		%\bibitem{bib:Kalitkin}
		%Калиткин~Н.\:Н. Численные методы.~М.:~Наука,~1978. 512~c.
		
		%\bibitem{bib:Berezin}
		%Березин~И.\:С., Жидков~Н.\:П. Методы вычислений. Том~1. Изд.~2-е, стереотип.~М.:~ГИЗ ФИЗМАТЛИТ,~1962. 464~c.
		
		%%% КНИГА ПОД РЕДАКЦИЕЙ
		%\bibitem{bib:Reibmann1}
		%Дисперсионная идентификация / Под ред. Н.\:С.~Райбмана.~М.:~Наука,~1981. 320~с.
		
		%%% СТАТЬЯ В СБОРНИКЕ
		%\bibitem{bib:Vlasov}
		%Власов~С.\:А., Шплихал~Й. Состояние разработок и перспективы развития имитационных систем для анализа функционирования и автоматизированного проектирования производства (на примерах металлургии и машиностроения) // Моделирование и идентификация 		производственных систем.~М.:~Институт проблем управления,~1988. С.~5~--~17.
		
		%\bibitem{bib:Reibmann2}
		%Райбман~Н.\:С. Методы нелинейной и минимаксной идентификации // Современные методы идентификации систем / Под ред. П.~Эйкхоффа.~М.:~Мир,~1983. С.~177~--~277.
		
		%\bibitem{bib:Bure:29}
		%Буре~В.\:М., Кирпичников~Б.\:К. Оптимальные решения по выборочным данным // Процессы управления и устойчивость: Труды 29-й научной конференции / Под ред. В.\:Н.~Старкова.~СПб.:~НИИ Химии СПбГУ,~1998. С.~296~--~299.
		
		%\bibitem{bib:Bure:32}
		%Буре~В.\:М. Кооперативное решение в задаче перестрахования риска // Процессы управления и устойчивость: Труды 32-й научной конференции студентов и аспирантов факультета ПМ-ПУ / Под ред. В.\:Н.~Старкова.~СПб.:~ООП НИИ Химии СПбГУ,~2001. С.~396~--~398.
		
		%%% ССЫЛКА НА ДОКУМЕНТ В ИНТЕРНЕТЕ		
		%\bibitem{bib:LAM} 
		%LAM/MPI Parallel Computing. http://en.wikibooks.org/wiki/LaTeX
		
		%\bibitem{bib:Latex}
		%LaTeX on Wikibooks. http://en.wikibooks.org/wiki/LaTeX
		
	\end{thebibliography}
    \endgroup
	%%\section{*Приложение}
	%%Приложи сюда свой код	
\end{document}