% Формат бумаги: А4.
\documentclass[a4paper]{report}
\usepackage[utf8]{inputenc}
\usepackage[russian]{babel}
\usepackage{amsmath,amsfonts,amssymb,cite,url,verbatim,graphicx,wrapfig}
\graphicspath{{images/}}

% Поля: верхнее – 2 см, нижнее – 2 см, левое – 3 см, правое – 1.5 см.
\usepackage{geometry}
\geometry{
	left = 3cm,
	top = 2cm,
	right = 1.5cm,
	bottom = 2cm
}


\setlength{\parindent}{1.5cm}

% Кегль: основной текст – 14 пт, названия параграфов – 16 пт, названия глав – 18 пт, текст в таблице, подписи к рисункам, таблицам – 12 пт.
\renewcommand{\small}{\fontsize{12}{12}\selectfont}
\renewcommand{\normalsize}{\fontsize{14}{14}\selectfont}
\renewcommand{\large}{\fontsize{16}{16}\selectfont}
\renewcommand{\Large}{\fontsize{18}{18}\selectfont}
\renewcommand{\huge}{\fontsize{20}{20}\selectfont}
\usepackage{sectsty}
\sectionfont{\Large}
\subsectionfont{\large}
\paragraphfont{\normalsize}
\usepackage[font=small]{caption}

% Межстрочный интервал: 1.5 строки.
\usepackage{setspace}
\onehalfspacing

% Абзацный отступ. Первая строка каждого абзаца должна иметь абзацный отступ 1.25 см.
\usepackage{indentfirst}
\setlength{\parindent}{1.25cm}

% Выравнивание основного текста по ширине поля.
\usepackage{ragged2e}
\justifying

%%% здесь лучше ничего не трогать
\setcounter{tocdepth}{2}
\renewcommand{\thesection}{}
\renewcommand{\thesubsection}{}
\allsectionsfont{\centering}
\usepackage{titlesec}
\newcommand{\sectionbreak}{\clearpage}
\usepackage{tocloft}
\cftsetindents{section}{0em}{0em}
\cftsetindents{subsection}{2em}{0em}
\AtBeginDocument{\renewcommand{\contentsname}{\begin{center} \vskip-2.5cm \Large{Содержание} \end{center}}}
\AtBeginDocument{\renewcommand{\bibname}{}}
%%%

\begin{document}
	%%%%% TITLE PAGE %%%%%
\thispagestyle{empty}
\begin{center}
\textsc{Санкт-Петербургский государственный университет \\
\textbf{Кафедра технологий программирования}} \\
\vspace{1cm}
\Large{\textbf{Башарин Егор Валерьевич}} \\
\vspace{1cm}
\large{\textbf{Выпускная квалификационная работа бакалавра}} \\ 
\vspace{2cm}
\Large{\textbf{Контекстный анализ данных социальных сетей}} \\ 
\normalsize{010400 \\
Прикладная математика и информатика} \\
\end{center}
\vspace{2cm}
\hspace{9cm} Научный руководитель, \\
\hspace*{9cm} ст. преподаватель \\
\hspace*{9cm} Попова С.В. 
\begin{center}
\vfill
Санкт-Петербург \\
2016
\end{center}
	\newpage
	
	%%%%% CONTENTS %%%%%
	\tableofcontents
	\newpage
	
	\section{Введение} 
	В настоящее время явление социальных сетей достаточно распространено. Социальные сети уверенно вошли в жизнь современного человека и теперь занимают в ней значимую часть. Главным образом они оказывают влияние на поведение, предубеждения, ценности и намерения человека, что отражается во всех сферах его деятельности. Оказываемое влияние, быстрый рост популярности и открытый доступ к контенту привлекли к социальным сетям внимание правительства, финансовых организации и исследователей. Преобразование контента социальных сетей в текстовую информацию и выделение ключевых концепций стало важным условием для порождения знаний и формулирования стратегий. Анализ полученной информации помогает исследователям улучшить понимание об информационных потоках, о формировании и распространении мнений, о связи ценностей и предубеждений пользователя и генерируемого им контента. Тем не менее число качественных и количественных исследований в данной области слишком мало. Самый значительный барьер при использовании социальных сетей - это отсутствие методологии для выбора, сбора, обработки и анализа информации, полученной с сайтов социальных сетей. Однако, существуют компании по производству программного обеспечения, разработавшие проприетарные системы сбора информации для визуализации данных, и исследователи, разработавшие экспертные системы для анализа настроений[1]. 
	
Пользователи социальных сетей активно публикуют данные о своей ежедневной активности, чувствах и мыслях, выражая свое мнение и позицию. Благодаря этому в социальных сетях образуются группы пользователей (сообщества), имеющих общие интересы. Для выявления ключевых концепций и тематик присущих группе пользователей используется контекстная обработка данных, генерируемых участниками группы. В данной работе контекстная обработка данных основана на идеях и принципах тематического моделирования. Результаты такой обработки данных могут использоваться для мониторинга заболеваний, например гриппа [2], или для предсказания поведения рынка. \\

	 
	\section{*Постановка задачи}
	Постановка твоей задачи. Постановка твоей задачи. Постановка твоей задачи. Постановка твоей задачи. Постановка твоей задачи. Постановка твоей задачи. Постановка твоей задачи. Постановка твоей задачи. Постановка твоей задачи. Постановка твоей задачи. Постановка твоей задачи. Постановка твоей задачи. Постановка твоей задачи. Постановка твоей задачи. 
	\section{*Обзор литературы}
	Обзор всякой разной литературы. Обзор всякой разной литературы. Обзор всякой разной литературы. Обзор всякой разной литературы. Обзор всякой разной литературы. Обзор всякой разной литературы. Обзор всякой разной литературы. Обзор всякой разной литературы. Обзор всякой разной литературы. Обзор всякой разной литературы. Обзор всякой разной литературы.  
	\section{*Глава 1. Это первая глава}
	Здесь надо больше всего написать. Здесь надо больше всего написать. Здесь надо больше всего написать. Здесь надо больше всего написать. Здесь надо больше всего написать. Здесь надо больше всего написать. Здесь надо больше всего написать. Здесь надо больше всего написать. Здесь надо больше всего написать. Здесь надо больше всего написать. Здесь надо больше всего написать. Здесь надо больше всего написать.
	\subsection{*1.1. Это первый параграф первой главы}
	Напиши здесь что-нибудь. Напиши здесь что-нибудь. Напиши здесь что-нибудь. Напиши здесь что-нибудь. Напиши здесь что-нибудь. Напиши здесь что-нибудь. Напиши здесь что-нибудь. Напиши здесь что-нибудь. Напиши здесь что-нибудь. Напиши здесь что-нибудь. Напиши здесь что-нибудь. Напиши здесь что-нибудь. Напиши здесь что-нибудь. Напиши здесь что-нибудь. Напиши здесь что-нибудь. Напиши здесь что-нибудь. Напиши здесь что-нибудь. Рис. \ref{fig:image1}
	
	%%Вставим картиночку и сошлёмся на неё. \ref{fig:image1}
	%%\begin{figure}[h]
	%%	\centering
	%%	\includegraphics[width=\textwidth]{eyeinjection.png}
	%%	\caption{Это подпись к картиночке \label%%%%%{fig:image1}}
	%%\end{figure}
	
	\section{*Выводы}
	Сделай свои выводы. Сделай свои выводы. Сделай свои выводы. Сделай свои выводы. Сделай свои выводы. Сделай свои выводы. Сделай свои выводы. Сделай свои выводы. Сделай свои выводы. Сделай свои выводы. Сделай свои выводы. Сделай свои выводы. Сделай свои выводы. 
	
	
	Сделай свои выводы. Сделай свои выводы. Сделай свои выводы. Сделай свои выводы. Сделай свои выводы. Сделай свои выводы. Сделай свои выводы. Сделай свои выводы. Сделай свои выводы. Сделай свои выводы. Сделай свои выводы. Сделай свои выводы. Сделай свои выводы. Сделай свои выводы. Сделай свои выводы. Сделай свои выводы. Рис. \ref{fig:image1:wrap}
	\section{*Заключение}
	Заключи что-нибудь. Заключи что-нибудь. Заключи что-нибудь. Заключи что-нибудь. Заключи что-нибудь. Заключи что-нибудь. Заключи что-нибудь. Таблица \ref{table1}
	
	%%% ДЛЯ СОЗДАНИЯ ТАБЛИЦ ОЧЕНЬ(!) РЕКОМЕНДУЮ 
	%%% http://www.tablesgenerator.com/
	
	\begin{table}[h]
		\begin{center}
			\begin{tabular}{ l c r }
			1 & 2 & 3 \\
			4 & 5 & 6 \\
			7 & 8 & 9 \\
			\end{tabular}
		\end{center}
		\caption{Это подпись к таблице \label{table1}}
	\end{table}
	
	Заключи что-нибудь. Заключи что-нибудь. Заключи что-нибудь. Заключи что-нибудь. Заключи что-нибудь. Заключи что-нибудь. Заключи что-нибудь. Заключи что-нибудь. Заключи что-нибудь. Заключи что-нибудь. Заключи что-нибудь. Заключи что-нибудь. Заключи что-нибудь. Заключи что-нибудь. Заключи что-нибудь. Заключи что-нибудь. Заключи что-нибудь. Заключи что-нибудь. Заключи что-нибудь. Заключи что-нибудь. Заключи что-нибудь. 
	
	\section{*Список литературы}
	
    \begingroup
    \let\clearpage\relax
    \vskip-3cm
	\begin{thebibliography}{9}
		%%% СТАТЬЯ В ЖУРНАЛЕ
		\bibitem{bib:Nadaraya} 
		Надарая~Э.\:А. Об оценке регрессии // Теория вероятностей и ее применения, 1964. Т.~9, Вып.~1. С.~157~--~159.
		
		\bibitem{bib:Sinitzin}
		Синицын~И.\:Н. Методы статистической линеаризации (обзор) // АиT,~1974. №~5. С.~82~--~94.
		
		\bibitem{bib:Billings}
		Billings~S.\:A., Fadzil~M.\:B., Sulley~J., Johnson~P.\:M. Identification of a nonlinear difference equation model of an industrial diesel generator // Mechanical Systems and Signal Processing,~1988. Vol.~2, No~1. P.~59~--~76.
		
		\bibitem{bib:Bouton}
		Booton~R.\:C. Nonlinear control systems with random inputs // Trans. IRE Profes. Group on Circuit Theory,~1954. Vol.~CT1, No~1. P.~9~--~18.
		
		\bibitem{bib:Boyd}
		Boyd~S., Chua~L.\:O. Fading memory and the problem of approximating nonlinear operators with Voltterra series // IEEE Trans. Circuits Syst.,~1985. Vol.~CAS-32, No~11. P.~1150~--~1161.
		
		\bibitem{bib:Gagarin}
		Garain~U. Identification of mathematical expressions in document images // Proc. of the 10th Int. Conf. on Document Analysis and Recognition (ICDAR),~2009. P.~1340~--~1344.
		
		\bibitem{bib:Lee}
		Lee~H.\:J., Wang~J.\:S. Design of a mathematical expression understanding system // Pattern Recognition Letters,~1997. Vol.~18, No~3, P.~289~--~298.
		
		%%% КНИГА ОДНОГО АВТОРА
		\bibitem{bib:Viner}
		Винер~В. Нелинейные задачи в теории случайных процессов.~М.:~ИЛ,~1961. 159~с.
		
		\bibitem{Steinberg}
		Штейнберг~Ш.\:Е. Идентификация в системах управления.~М.:~Энергоатомиздат,~1987. 80~с.
		
		\bibitem{bib:Petrov}
		Петров~Л.\:И. О культуре набора текста на ПМ–ПУ.~СПб:~Изд. дом C.-Петерб. ун-та,~1726. 324~c.
		
		%%% КНИГА НЕСКОЛЬКИХ АВТОРОВ
		\bibitem{bib:Demidovich}
		Демидович~Б.\:П., Марон~И.\:А., Шувалова~Э.\:З. Численные методы анализа. Приближение функций, дифференциальные и интегральные уравнения.~М.:~Наука,~1967. 368~c.
		
		\bibitem{bib:Kalitkin}
		Калиткин~Н.\:Н. Численные методы.~М.:~Наука,~1978. 512~c.
		
		\bibitem{bib:Berezin}
		Березин~И.\:С., Жидков~Н.\:П. Методы вычислений. Том~1. Изд.~2-е, стереотип.~М.:~ГИЗ ФИЗМАТЛИТ,~1962. 464~c.
		
		%%% КНИГА ПОД РЕДАКЦИЕЙ
		\bibitem{bib:Reibmann1}
		Дисперсионная идентификация / Под ред. Н.\:С.~Райбмана.~М.:~Наука,~1981. 320~с.
		
		%%% СТАТЬЯ В СБОРНИКЕ
		\bibitem{bib:Vlasov}
		Власов~С.\:А., Шплихал~Й. Состояние разработок и перспективы развития имитационных систем для анализа функционирования и автоматизированного проектирования производства (на примерах металлургии и машиностроения) // Моделирование и идентификация 		производственных систем.~М.:~Институт проблем управления,~1988. С.~5~--~17.
		
		\bibitem{bib:Reibmann2}
		Райбман~Н.\:С. Методы нелинейной и минимаксной идентификации // Современные методы идентификации систем / Под ред. П.~Эйкхоффа.~М.:~Мир,~1983. С.~177~--~277.
		
		\bibitem{bib:Bure:29}
		Буре~В.\:М., Кирпичников~Б.\:К. Оптимальные решения по выборочным данным // Процессы управления и устойчивость: Труды 29-й научной конференции / Под ред. В.\:Н.~Старкова.~СПб.:~НИИ Химии СПбГУ,~1998. С.~296~--~299.
		
		\bibitem{bib:Bure:32}
		Буре~В.\:М. Кооперативное решение в задаче перестрахования риска // Процессы управления и устойчивость: Труды 32-й научной конференции студентов и аспирантов факультета ПМ-ПУ / Под ред. В.\:Н.~Старкова.~СПб.:~ООП НИИ Химии СПбГУ,~2001. С.~396~--~398.
		
		%%% ССЫЛКА НА ДОКУМЕНТ В ИНТЕРНЕТЕ		
		\bibitem{bib:LAM} 
		LAM/MPI Parallel Computing. http://en.wikibooks.org/wiki/LaTeX
		
		\bibitem{bib:Latex}
		LaTeX on Wikibooks. http://en.wikibooks.org/wiki/LaTeX
		
	\end{thebibliography}
    \endgroup
	\section{*Приложение}
	Приложи сюда свой код	
\end{document}